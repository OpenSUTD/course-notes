\documentclass[a4paper]{article}

\usepackage[margin=1in]{geometry} 
\usepackage[hidelinks]{hyperref}
\usepackage{tocloft}
\renewcommand{\cftsecleader}{\cftdotfill{\cftdotsep}}
\renewcommand{\baselinestretch}{1.25} 

\usepackage{minted}
\usepackage{inconsolata}

\usepackage{csquotes}

\usepackage{enumitem}

\usepackage{booktabs}% http://ctan.org/pkg/booktabs
\newcommand{\tabitem}{~~\llap{\textbullet}~~}

\usepackage{graphicx}
\graphicspath{ {./images/} }

\setcounter{tocdepth}{4}
\setcounter{secnumdepth}{3}

\usepackage{amssymb}
\usepackage{amsmath}

\usepackage{array}

\usepackage{cellspace}
\setlength\cellspacetoplimit{4pt}
\setlength\cellspacebottomlimit{4pt}

\newcommand\cincludegraphics[2][]{\raisebox{-0.3\height}{\includegraphics[#1]{#2}}}

\usepackage{textcomp}
\newcommand{\textapprox}{\raisebox{0.5ex}{\texttildelow}}

\title{50.005 Computer System Engineering}
\author{Tey Siew Wen}

\date{15 Oct 2019}
\begin{document}
\maketitle
	
\tableofcontents
	
\newpage

\section{Networking}
\subsection{The Internet - A nuts \& bolts view}
\begin{enumerate}
	\item Hosts (End Systems) running network apps
	\item Communication Links
	\item Packet Switches
\end{enumerate}	
\subsection{The Packet}
\begin{itemize}
	\item Packet = Header + Payload
	\begin{itemize}[label=$\circ$]
		\item Header data is used by networking hardware to direct the packet to its destination
		\item Payload is extracted and used by application software.
	\end{itemize}
	\item Usually, the mode of transmission of packets is: PC $\rightarrow$ Switch $\rightarrow$ Router $\rightarrow$ Modem $\rightarrow$ Modem $\rightarrow$ Switch $\rightarrow$ Servers
	\item When the packets arrive at the router, the packet arrival rate to link temporarily exceeds the output link capacity. Hence they would queue in the router buffers, causing a delay.
	\begin{itemize}[label=$\circ$]
		\item If there are no free buffers for them to get into queue, then the packets will be dropped. This is known as packet loss.
	\end{itemize}
\end{itemize}
\subsubsection{Single Router Packet Delay Sources}
$$d_{nodal} = d_{proc} + d_{queue} + d_{trans} + d_{prop}$$
\begin{enumerate}
	\item Nodal Processing $(d_{nodal})$
	\begin{itemize}[label=$\circ$]
		\item Examine packet header
		\item Check for bit errors
		\item Determine output link (destination IP address in packet header)
	\end{itemize}
	\item Queueing $(d_{queue})$
	\begin{itemize}[label=$\circ$]
		\item Waiting time for packet to get to front of queue for output link
		\item Depends on congestion level (how much other users are also sending data)
	\end{itemize}
	\item Transmission $(d_{trans})$
	\begin{itemize}[label=$\circ$]
		\item Time to push the whole packet (all the bits) from router to link $$ d_{trans} = \frac{\text{L}}{\text{R}} = \frac{\text{packet length (bits)}}{\text{link bandwith (bps)}}$$
	\end{itemize}
	\item Propagation $(d_{prop})$
	\begin{itemize}[label=$\circ$]
		\item Time for packet to move from beginning to end of the link,
		$d_{prop} = \frac{\text{length of physical link (m)}}{\text{propagation speed in medium}}$
		\item Propagation speed in medium is $\approx 2\times10^{8}$ m/s ($\frac{2}{3}$ speed of light in vacuum)
		$$ d_{prop} = \frac{\text{d}}{\text{s}} \text{m/s} $$
	\end{itemize}
\end{enumerate}
\subsubsection{End-to-end Delay for many hosts}
\begin{itemize}
	\item If not given space-time diagram $$ d_{end} = \text{no. of hops} \times \text{nodal delay} $$
	\item If given space-time diagram, we can approximate $d_{end}$ from the graph.
\end{itemize}
\paragraph{Calculating Throughput}\mbox{}\\
Throughput: File Receival rate (bits/s)
\begin{itemize}
	\item Instantaneous Throughput : at any instant of time
	\item Average Throughput : over a longer period
	\begin{itemize}[label=$\circ$]
		\item for a file consisting of F bits
		\item and the transfer takes T seconds for Host B to receive all F bits, then the calculation for average throughput is as shown below. $$ \frac{\text{F}}{\text{T}} \ \text{bits/s}$$
	\end{itemize}
	\item Effective throughput is determined by the slowest bandwidth in the route, the \textbf{\textit{bottleneck link}}.
\end{itemize}
\subsubsection{Average Link Utilization}
Also known as \textit{Traffic Intensity}, given by the formula
$$ \text{Traffic Intensity} = \frac{\text{La}}{\text{R}} $$
where:
\begin{itemize}
	\item $L$ is the packet length (bits)
	\item $a$ = average packet arrival rate
	\item $R$ = link bandwidth (bits/s)
\end{itemize}
\mbox{}\\
If $\frac{\text{La}}{\text{R}} > 1$,
\begin{itemize}
	\item the average rate at which bits arrive at the queue exceeds the rate at which the bits can be transmitted from the queue.
	\item the queue will tend to increase without bound and the queuing delay will approach infinity.
	\item so system must be designed that traffic intensity should be $\leq$ 1.
\end{itemize}
\mbox{}\\
If $\frac{\text{La}}{\text{R}} \leq 1$,
\begin{itemize}
	\item If packets arrive individually periodically, then every packet will arrive at an empty queue and there will be no queueing delay
	\item If the packets arrive periodically in bursts (bursty data), then the nth packet transmitted will have a queueing delay of $$(n-1)\times\frac{\text{L}}{\text{R}} \text { seconds}$$
\end{itemize}
\begin{framed}
	\begin{displayquote}
		Do note that the cases stated here are more academic examples than reality. In reality,\\ typically, the arrival process to a queue is random; that is, the arrivals do not follow any\\pattern and the packets are spaced apart by random amounts of time. $\frac{\text{La}}{\text{R}}$ is not usually\\ sufficient to fully characterize the queueing delay statistics.
	\end{displayquote}
\end{framed}
\subsubsection{Modems}
\begin{itemize}
	\item Carries out Modulation \& Demodulation
	\item Modulation: A process to add text/img/vid/sound etc into a carrier signal. Demodulation is vice versa.
	\begin{itemize}[label=$\circ$]
		\item Types of broadbands:
		\begin{itemize}[label=\tiny$\blacksquare$]
			\item Dial-up: audio freq $\leftrightarrow$ digital
			\item DSL: copper telephone lines
			\item Coaxial Cables: faster than DSL
			\item Fibre Optic Cables: Light pulse
		\end{itemize}
		s
		Wifi signals: EM waves
	\end{itemize}
	\item But for modern modems, there is no longer a need for conversion of signals from audio freq $\leftrightarrow$ digital. Their primary role is for converting electrical signals over long connections, into some signal form that is understood by router \& computers.
\end{itemize}
\subsection{The Internet - A service view}
\begin{enumerate}
	\item Protocols for communication
	\begin{itemize}[label=$\circ$]
		\item Define format, order of messages sent \& received
		\item Appropriate actions to be taken on message transmission and receipts.
		\item Set by RFC (Request for Comment) documents, published by IETF (Internet Engineering Task Force).
		\item e.g. TCP connection request $\rightarrow$ connection response $\rightarrow$ Get request to a server URI $\rightarrow$ receive packet for a file
	\end{itemize}
	\item Network API for infrastructures to \textbf{provide services to user applications}
	\item \textbf{Sharing of info among many devices} via circuit switching \& packet switching
	\item \textbf{Scalability of systems} via the Hierarchy of ISPs
\end{enumerate}
\newpage
\subsection{Internet Protocol Stack (Layering)}
\begin{enumerate}
	\item Application: messages
	\item Transport: segments
	\item Network: datagrams
	\item Link: frames
	\item Physical: bits
\end{enumerate}
The stack is separated into 5 layers so that we can \textbf{avoid complex interactions between\\ components}.
\begin{itemize}
	\item The packet's header \& payload is relative to the layer.
	\item E.g. Transport layer's header is network layer's payload.
	\item Number of possible interactions with N layers is N-1.
	\item Without layers, there are $\text{N}^2$ possible interactions instead.
	\begin{itemize}[label=$\circ$]
		\item Difficult to debug
		\item Might lead to emergent \& undesirable behaviour
	\end{itemize}
\end{itemize}
\subsubsection{Internet Control Message Protocol (ICMP)}
ICMP is used by hosts and routers to communicate network-layer information to each other. ICMP is often considered part of IP but architecturally it lies just above IP, as \textbf{ICMP messages are carried inside IP datagrams}. (\textbf{is also carried as IP payload}) \textit{Similar to TCP/UDP segments being carried as IP payload}.\\
\\ When a host receives an IP datagram with ICMP specified as the upper-layer protocol, it demultiplexes the datagram’s contents to ICMP.\\
\\ICMP messages have a type and a code field, and contain the header and the first 8 bytes of the IP datagram that caused the ICMP message to be generated in the first place.The sender can then determine the datagram that caused the error.\\
\\
Types of ICMP Messages:
\begin{itemize}
	\item Source Quench: perform congestion control - congested routers send this message to a host to force that host to reduce its transmission rate
	\item Port Unreachable
	\item Ping Echo request/reply
\end{itemize}
\newpage
\subsubsection{Internet Service Provider (ISP)}
An organization that provides services for accessing and using the internet. It is also a network of packet switches \& comm links.\\
\\
The hierarchy of ISPs: 
\begin{center}
	\includegraphics[scale=0.6,trim={0 0 0 0.5cm},clip]{isp.png}
\end{center}
\begin{itemize}
	\item IXP connects the various global ISPs together.
	\item Then the small individual access ISP subscribe to commercial global Tier 1 ISPs.
	\item Google here is not a global ISP, but rather a Content Delivery Network (CDN), which is a private network that carries data to \& from Google servers only, but connects to everyone.
\end{itemize}
\newpage
\subsection{Sharing of Information via Switching}
\subsubsection{Circuit Switching}
Circuit switching can performed via Time Domain Multiplexing (TDM) \& Frequency Domain Multiplexing (FDM).\\
\\
\textbf{Multiplexing} is the process of combining multiple signals into one, in such a manner that each individual signal can be retrieved at the destination. These multiple signals are also travelling in the same channel.
\begin{table}[H]
	\centering
	\begin{tabular}{|l|l|}
		\hline
		\multicolumn{1}{|c|}{\textbf{TDM}}                                                                                                                                                                                                             & \multicolumn{1}{c|}{\textbf{FDM}}                                                                                                                                                                                                                                       \\ \hline
		Bandwidth = bits/second                                                                                                                                                                                                                        & Bandwidth = frequency range                                                                                                                                                                                                                                             \\ \hline
		\begin{tabular}[c]{@{}l@{}}Divides and allocates certain time periods \\ to each channel in an alternating manner.\end{tabular}                                                                                                                & \begin{tabular}[c]{@{}l@{}}Divides the channel into two or more\\ frequency ranges that do not overlap.\end{tabular}                                                                                                                                                    \\ \hline
		\begin{tabular}[c]{@{}l@{}}Greater flexibility as it can dynamically \\ allocate more time periods to the signals \\ that need more of the bandwidth, while \\ reducing the time periods to those \\ signals that do not need it.\end{tabular} & \begin{tabular}[c]{@{}l@{}}Less flexible since it preallocates use \\ of transmission link regardless of demand. \\ Cannot dynamically change the width of \\ the allocated frequency. So there could be \\ allocated but unneeded link time going unused.\end{tabular} \\ \hline
		\begin{tabular}[c]{@{}l@{}}Each signal uses all of the bandwidth \\ some of the time.\end{tabular}                                                                                                                                             & \begin{tabular}[c]{@{}l@{}}Each signal uses a small portion of the\\ bandwidth all of the time.\end{tabular}                                                                                                                                                            \\ \hline
	\end{tabular}
\end{table}
\noindent While there are advantages \& disadvantages to each method, FDM and TDM are \textit{often used in tandem}, to create even more channels in a given frequency range.\\
\\
A common practice for telecoms to allow a huge number of users to use a certain frequency band:
\begin{itemize}
	\item \textbf{divide the channel with FDM}, so that you have a dedicated channel with a smaller frequency range.
	\item Each of the FDM channels is then \textbf{occupied by multiple channels that are multiplexed using TDM}.
\end{itemize}
Circuit switching allows a \textbf{fixed, dedicated fraction of the link for each user}. Better for continuous, streamline usage.
\subsubsection{Packet Switching}
On the other hand, Packet switching occupy link on demand, \textbf{better for bursty data}. e.g. A link shared among many users, but only a low proportion of users are active at the same time.
\newpage
\section{Operating System (OS)}
\textbf{Resource allocation}
\begin{itemize}
	\item Policy: Manages all resources
	\item Decides between conflicting requests (efficiency and fairness)
\end{itemize}
\textbf{Control program}
\begin{itemize}
	\item Mechanism: Controls execution of programs (prevent errors and misuse)
	\item Modern operating system is interrupt driven. The operating system preserves the state of the CPU by storing registers and the program counter.
\end{itemize}
\subsection{Kernel}
\begin{itemize}
	\item runs with special privileges – it can do what normal user code cannot do (e.g., access to special instructions \& special memory regions)
	\item Hardware support required – CPU has (at least) dual mode operation: user (unprivileged) mode vs. kernel (privileged) mode
	\item System calls let (unprivileged, untrusted) user programs access (privileged, trusted) kernel services
	\item System call changes mode to kernel, return from call (via return-from-trap or RETT instruction) resets it to user
\end{itemize}
\begin{table}[H]
	\centering
	\begin{tabular}{|l|l|l|l|}
		\hline
		\multicolumn{1}{|c|}{\textbf{Kernel type}} & \multicolumn{1}{c|}{\textbf{Idea}}                                       & \multicolumn{1}{c|}{\textbf{Benefits}}                                       & \multicolumn{1}{c|}{\textbf{Detriments}}                                                                      \\ \hline
		Monolithic                                 & Simple layered structure                                                 & ?                                                                            & ?                                                                                                             \\ \hline
		Microkernel                                & \begin{tabular}[c]{@{}l@{}}Move services into \\ user space\end{tabular} & \begin{tabular}[c]{@{}l@{}}Easier development, \\ more reliable\end{tabular} & \begin{tabular}[c]{@{}l@{}}Performance overhead of communication\\ between user and kernel space\end{tabular} \\ \hline
		Hybrid                                     & ?                                                                        & ?                                                                            & ?                                                                                                             \\ \hline
	\end{tabular}
\end{table}
\textbf{Dual-mode Operation}
\begin{itemize}
	\item Software error or request creates exception or trap (software interrupt)
	\item Dual-mode operation: User mode and (privileged) kernel mode
	\item User-mode program is usually limited to its own address space so that it cannot access or modify other running programs or the operating system itself, and is usually prevented from directly manipulating hardware devices
	\item Mode bit provided by hardware provides ability to distinguish when system is running user code or kernel code
	\item System program is in user mode!
	\item System call: API to services provided by OS kernel
	\item System calls are made available by the operating system to provide well-defined, safe implementations for operations that require access to e.g. hardware devices
\end{itemize}
\newpage
\noindent \textbf{OS Services}
\begin{itemize}
	\item Communications between processes (IPC)
	\item Error and misuse detection: Owners of information stored in a multiuser or networked computer system may want to control use of that information; concurrent processes should not interfere (logically) with each other
\end{itemize}
\subsubsection{Computer-system operation}
\begin{itemize}
	\item One or more CPUs, device controllers connect through common bus providing access to shared memory and other resources
\end{itemize}
\subsection{Processes}
\begin{itemize}
	\item Process is a program in execution. It is a unit of work within the system. Program is a passive entity, process is an active entity which requires resources to complete a task
	\item Process contains a PC, stack, data section and allocated memory in heap
	\item Process defines a private address space
	\item Process couples abstractions such as concurrency and protection
	\item Process identified and managed via \texttt{int pid} process identifier
	\item Single-threaded process has one program counter specifying location of next instruction to execute, while multi-threaded process has one program counter per thread
\end{itemize}
\begin{table}[H]
	\centering
\begin{tabular}{|l|l|}
	\hline
	\multicolumn{1}{|c|}{\textbf{Process State}} & \multicolumn{1}{c|}{\textbf{Explanation}} \\ \hline
	new                                          & being created                             \\ \hline
	running                                      & executing on CPU                          \\ \hline
	waiting/blocked                              & wait for I/O or event                     \\ \hline
	ready                                        & queue for execution                       \\ \hline
	terminated/stop                              & finished execution                        \\ \hline
\end{tabular}
\end{table}
Process control block (PCB) is a data structure that contains all information about active processes.
\begin{itemize}
	\item Job queue: set of all processes
	\item Ready queue: processes in RAM that are ready to execute
	\item Device queue: waiting for I/O
\end{itemize}
\subsubsection{Process Management}
\begin{itemize}
	\item Cooperating processes may affect each other, mainly through sharing data
	\item Two basic problems: \textbf{mutual exclusion} and \textbf{condition synchronization}
	\item Two basic models of IPC: Shared memory and Message passing
	\item Socket as an example of message passing
	\item Thread: line of execution, has registers + stack
	\item Many threads can run within a process, sharing the address space
	\item Kernel Thread
	\begin{itemize}[label=$\circ$]
		\item known to OS and scheduled by kernel
		\item represented within a kernel-specific data structure
	\end{itemize}
	\item User Thread
	\begin{itemize}[label=$\circ$]
		\item not known to OS kernel
		\item scheduled by user-mode thread scheduler (such as \texttt{pthread})
		\item must belong to a process
	\end{itemize}
\end{itemize}
Because a thread is smaller than a process, thread creation typically uses fewer resources than process creation. Creating a process requires allocating a process control block (PCB), a rather large data structure. The PCB includes a memory map, list of open files, and environment variables. Allocating and managing the memory map is typically the most time-consuming activity.
\paragraph{Distribution Systems}\mbox{}\\
\\
\textbf{Single-core}
\begin{itemize}
	\item Concurrency: CPU performs context switching so frequently that it gives the illusion of multi-tasking and users can interact with each job while it is running
\end{itemize}
\noindent \textbf{Multi-processor}
\begin{itemize}
	\item Multiprocessing overheads like context switching, IPC, or inter-process synchronization. Hence the speed-up in reality will never be the theoretical maximum of \texttt{N}
	\item Context switching in user threads us faster because it doesn’t have to trap to the kernel (no system call overhead)
\end{itemize}
\subsubsection{Producer-Consumer Problem}
\begin{itemize}
	\item producer puts a new item of work into a shared buffer
	\item consumer takes an item of work from the buffer
	\item buffer can store a fixed number of items (bounded buffer)
\end{itemize}
\subsubsection{Critical Section Problem}
\begin{itemize}
	\item Mutual Exclusion - If process is executing in its critical section (CS), then no other processes can be executing in their critical sections
	\item Progress - If no process is executing in its critical section and there exist some processes that wish to enter their critical section, then the selection of the processes that will enter the critical section next cannot be postponed indefinitely
	\item Bounded Waiting - A bound must exist on the number of times that other processes are allowed to enter their critical sections after a process has made a request to enter its critical section and before that request is granted
\end{itemize}
\paragraph{Peterson's Solution}
\begin{itemize}
	\item require busy waiting, generally bad for uniprocessors
	\item busy waiting (spinlock) means executing instructions (polling the lock) without doing anything useful
\end{itemize}
\paragraph{Synchronization Hardware}
\begin{itemize}
	\item Many modern machines provide special atomic hardware instructions
	\item Test original value of memory word and set its value
	\item Swap two memory words
\end{itemize}
\paragraph{Semaphore}
\begin{itemize}
	\item Semaphore is a high-level synchronization primitive
	\item No busy waiting
	\item Two atomic operations on the \texttt{int} \textbf{state variable}
	\item \texttt{acquire()} ++ and \texttt{release()}
	\item These are critical sections, hence require hardware instructions or use of spinlock
	\item Binary (0/1) semaphore or Counting semaphore (\texttt{int})
\end{itemize}
\paragraph{Java Synchronization}
\begin{itemize}
	\item Each Java object has an associated binary lock
	\item Lock is acquired/released by invoking a synchronized method
	\item Mutual exclusion is guaranteed for this object’s method – at most only one thread can be inside it at any time
	\item Threads waiting to acquire the object lock are placed in the \textbf{entry set} for the object lock
	\item A synchronized method of an object can successfully call another synchronized method of the same object although both synchronized methods are guarded by the same binary lock
\end{itemize}
\subsubsection{Deadlocks}
\begin{itemize}
	\item Each process utilizes a resource as follows: request, use, release
	\item \textbf{Circular wait}: set of blocked processes each holding a resource and waiting to acquire a resource held by another process in the set
	\item Single lane bridge crossing example: if a deadlock occurs, it can be resolved if one car backs up (pre-empt resources and rollback)
	\item Cycle in resource allocation graph: if resource only has one instance, then \textbf{deadlock}, else, possibility of deadlock
\end{itemize}
Necessary, but not sufficient, for deadlock: all four conditions hold \textbf{simultaneously}:
\begin{table}[H]
	\centering
	\begin{tabular}{|l|l|}
		\hline
		\multicolumn{1}{|c|}{\textbf{Condition}} & \multicolumn{1}{c|}{\textbf{Elaboration}}                                                                                                                      \\ \hline
		Mutual exclusion                         & Only one process can use a resource                                                                                                                            \\ \hline
		Hold and wait                            & \begin{tabular}[c]{@{}l@{}}One process holds a resource while \\ waiting for other resources\end{tabular}                                                      \\ \hline
		No preemption                            & Resources are only release voluntarily                                                                                                                         \\ \hline
		Circular wait                            & \begin{tabular}[c]{@{}l@{}}There exists a set of waiting processes \\ such that one is waiting for another that \\ is held by yet another process\end{tabular} \\ \hline
	\end{tabular}
\end{table}
\noindent Pre-emptive scheduling allows a process to be interrupted in the midst of its execution, taking the CPU away and allocating it to another process. Non-pre-emptive scheduling ensures that a process relinquishes control of the CPU only when it finishes with its current CPU burst.
\paragraph{Deadlock avoidance}
\begin{itemize}
	\item Before granting a resource request (even if request is valid and the requested resources are now available), check that the request will leave system in a \textbf{safe state}
	\item By right not possible for system to enter an unsafe state, but if it does there is possibility of deadlock
	\item Requires advance knowledge of future resource needs (e.g. Banker’s algorithm)
\end{itemize}
\paragraph{Deadlock detection and recovery}
\begin{itemize}
	\item Actively detect deadlock after the fact, then recover from it (e.g. pre-empting held resources and rolling back processes)
\end{itemize}
\paragraph{Deadlock prevention}
\begin{itemize}
	\item Impose conditions on resource requests to ensure that a valid request can never cause the system to enter a deadlock state by design without having to check the system's detailed resource allocation state
	\begin{itemize}[label=$\circ$]
		\item basically disallow any of the four necessary conditions for deadlock
	\end{itemize}
	\item No resource hold and wait
	\begin{itemize}[label=$\circ$]
		\item Must get all resources before process execution
		\item Only allow request for resources if the process has none
	\end{itemize}
	\item Disallow circular wait
	\item Enforces pre-emption
	\begin{itemize}[label=$\circ$]
		\item Process must release all resources its already holding if it needs another resources that require wait
		\item Restart process, must wait for every resources again
	\end{itemize}
\end{itemize}
\subsubsection{Banker's Algorithm}
Data structure
\begin{itemize}
	\item Available: 1D array
	\item Max: \texttt{n*m} 2D array
	\item Allocation: \texttt{n*m} 2D array
	\item Need: \texttt{n*m} 2D array
\end{itemize}
\noindent Safety Algorithm
\begin{enumerate}
	\item Let Work and Finish be vectors of length \texttt{m} and \texttt{n}, respectively
	\begin{itemize}[label=$\circ$]
		\item \texttt{n} = number of processes, \texttt{m} = number of resource types
		\item \texttt{work = available}
		\item \texttt{finish[i] false for i in {0 .. n-1}}
	\end{itemize}
	\item Find \texttt{i} such that
	\begin{itemize}[label=$\circ$]
		\item \texttt{ finish[i] == false}
		\item \texttt{need[i]} $\leq$ work
		\item else: skip to Step 4
	\end{itemize}
	\item \texttt{work = work + allocation[i]; finish[i] = true; GOTO step 2}
	\item if \texttt{finish[i] == true} for all \texttt{i}, then only system is in a safe state
\end{enumerate}
\subsubsection{Interrupt Handling}
\begin{itemize}
	\item Interrupt handling is done in \textbf{kernel mode}, by the service routine
	\item Interrupts transfers control to the interrupt service routine generally, through the \textbf{interrupt \\ vector}, which contains the addresses of all the service routines
	\item Interrupt architecture must save the address of the interrupted instruction. Incoming interrupts are disabled while another interrupt (of same or higher priority) is being processed to prevent a lost interrupt or reentrancy problems
	\item \textbf{Context switching}: need to save/restore into PCB all the CPU hardware state that keeps track of the process’s execution, e.g., registers like SP, PC, PSR, etc. to/from main memory
	\item OS Scheduler interrupts periodically running processes give control to the OS, which can then force rescheduling even if the process doesn’t give up the CPU voluntarily
	\item \textbf{trap} is software-generated interrupt caused either by an error or a user request to allow a user program to invoke an OS function (system call) and run it in kernel mode
	\item Polling interrupt checks each I/O devices one by oneon whether this is the correct device that requests for I/O interrupt
	\item Vectored interrupt identifies the device that sent the request
\end{itemize}
\subsection{Storage Hierarchy}
\begin{table}[H]
	\centering
	\begin{tabular}{|l|c|c|c|}
		\hline
		\multicolumn{1}{|c|}{\textbf{System}} & \textbf{Cost} & \textbf{Speed} & \textbf{Capacity} \\ \hline
		Registers                             & highest       & highest        & \textless  \ KB      \\ \hline
		L(X) Cache                            &               &                & \textgreater \ MB   \\ \hline
		RAM                                   &               &                & \textgreater \ GB   \\ \hline
		NVRAM                                 &               &                & \textgreater \ GB   \\ \hline
		Disk                                  & lowest        & slowest        & \textgreater \ TB   \\ \hline
	\end{tabular}
\end{table}
\begin{itemize}
	\item Caching: Information in use copied from slower to faster storage temporarily
	\item If processes don’t fit in memory, swapping moves them in and out to run
\end{itemize}
\subsubsection{File System}
\begin{table}[H]
	\centering
	\begin{tabular}{|l|l|}
		\hline
		\multicolumn{1}{|c|}{\textbf{UNIX cmd}} & \multicolumn{1}{c|}{\textbf{function}} \\ \hline
		\texttt{cat}                                     & display file contents                  \\ \hline
		\texttt{rm}                                      & remove file                            \\ \hline
		\texttt{mkdir}/\texttt{rmdir}                             & make/remove dir                        \\ \hline
		\texttt{cd}                                      & change current dir                     \\ \hline
		\texttt{find }                                   & traverse and find                      \\ \hline
		\texttt{ln source link  }                        & create hard link                       \\ \hline
		\texttt{ln -s source link}                       & create symbolic link                   \\ \hline
	\end{tabular}
\end{table}
\paragraph{Files}
\begin{itemize}
	\item Files are stored on persistent (non-volatile) storage, and in UNIX are memory cached for \\performance:
	\begin{itemize}[label=$\circ$]
		\item regular files
		\item special files:
		\begin{itemize}[label=\tiny$\blacksquare$]
			\item folders
			\item \texttt{.} refers to current directory
			\item \texttt{..} refers to parent directory
		\end{itemize}
		\item file links:
		\begin{itemize}[label=\tiny$\blacksquare$]
			\item symbolic link (shortcut link)
			\item hard link (both path point to the same file)
		\end{itemize}
	\end{itemize}
	\item file is an uninterpreted sequence of words/bytes
	\item can be viewed as an abstract data type, like a class of objects in the OOP sense
	\begin{itemize}[label=$\circ$]
		\item state: file data and metadata (file attributes)
		\item interface (set of methods e.g. open/rw/del to be used)
	\end{itemize}
	\item format is interpreted by some application (or OS kernel in case of special files)
	\item UNIX doesn’t allow hard links to directories. This restriction ensures that we can’t form cycles with hard links 
\end{itemize}
\paragraph{File Descriptor Tables}
\begin{itemize}
	\item 2 file descriptors \texttt{fd1} and \texttt{fd2} reference same open file entry, they share usage (e.g., cp) of the file – read/write through fd1 will advance cp seen by fd2
\end{itemize}
\paragraph{Filesystem Structure}
\begin{itemize}
	\item Directory Structure is meta-data that organizes files in a structured name space
	\item Both the directory structure and the files themselves reside on disk (and cached in memory)
	\item There are three types of directory structures:
	\begin{enumerate}[label=\roman*.]
		\item Tree - no links, one location in filesystem for every inode location
		\item Acyclic graph (UNIX) - possible to have multiple links/path to one inode location
		\item General graph - possible to have cycles of links
	\end{enumerate}
\end{itemize}
\end{document}